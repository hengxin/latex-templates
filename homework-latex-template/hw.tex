% hw.tex: LaTeX template for homework
% Feel free to modify it (mainly the 'preamble' file).

\documentclass[11pt, a4paper, UTF8]{ctexart}
% File: preamble.tex
% left = 3cm, right = 2cm,
\usepackage[top = 1.5cm]{geometry}

% Set fonts commands
\newcommand{\song}{\CJKfamily{song}} 
\newcommand{\hei}{\CJKfamily{hei}} 
\newcommand{\kai}{\CJKfamily{kai}} 
\newcommand{\fs}{\CJKfamily{fs}}

\newcommand{\me}[2]{\author{{\bfseries 姓名:}\underline{#1}\hspace{2em}{\bfseries 学号:}\underline{#2}}}

% Always keep this.
\newcommand{\noplagiarism}{
  \begin{center}
    请独立完成作业,不得抄袭。\\
    若参考了其它资料,请给出引用。\\
    鼓励讨论,但需独立书写解题过程。
  \end{center}
}

% Each hw may consist of three parts:
% (1) this homework
\newcommand{\beginthishw}{\part{作业}}
% (2) corrections
\newcommand{\begincorrection}{\part{订正}}
% (3) any feedback
\newcommand{\beginfb}{\part{反馈}}

% For math
\usepackage{amsmath}
\usepackage{amsfonts}
\usepackage{amssymb}

% Define theorem-like environments
\usepackage[amsmath, thmmarks]{ntheorem}

\theoremstyle{break}
\theorembodyfont{\song}
\theoremseparator{}
\newtheorem*{problem}{题目}

\theorempreskip{2.0\topsep}
\theoremheaderfont{\kai\bfseries}
\theoremseparator{:}
\newtheorem*{remark}{注}
\theorempostwork{\bigskip\hrule}
\newtheorem*{solution}{解答}
\theorempostwork{\bigskip\hrule}
\newtheorem*{revision}{订正}

\theoremstyle{plain}
\newtheorem*{cause}{错因分析}

\theoremstyle{break}
\theorempostwork{\bigskip\hrule}
\theoremsymbol{\ensuremath{\Box}}
\newtheorem*{proof}{证明}

\renewcommand\figurename{图}
\renewcommand\tablename{表}

% For figures
% for fig with caption: #1: width/size; #2: fig file; #3: fig caption
\newcommand{\fig}[3]{
  \begin{figure}[htp]
    \centering
      \includegraphics[#1]{#2}
      \caption{#3}
  \end{figure}
}

% for fig without caption: #1: width/size; #2: fig file
\newcommand{\fignocaption}[2]{
  \begin{figure}[htp]
    \centering
    \includegraphics[#1]{#2}
  \end{figure}
}


%%%%%%%%%%%%%%%%%%%%
\title{第一讲:线性规划}
\me{姓名}{学号}
\date{\today}
%%%%%%%%%%%%%%%%%%%%
\begin{document}
\maketitle
%%%%%%%%%%%%%%%%%%%%
\beginthishw	%	begin this homework (hw) 

%%%%%%%%%%
\begin{problem}[29.1-4]	% note: use '[]' for optional information
  A problem here.
\end{problem}

% Important Note: Now, this ``remark'' environment should be used before
% the ``solution''/``revision''/``proof'' environment.
\begin{remark}	
  Refer to book.
\end{remark}

\begin{solution}
  Standard form 如下:
\end{solution}
%%%%%%%%%%
\begin{problem}[xx.x-x]
  假设这是一道需要证明的题目。	
\end{problem}

% Important Note: Now, this ``remark'' environment should be used before
% the ``solution''/``revision''/``proof'' environment.
% \begin{remark}	
%   Refer to book.
% \end{remark}

\begin{proof}
  证明略。	
\end{proof}
%%%%%%%%%%
%%%%%%%%%%%%%%%%%%%%
\beginlasthw	% begin the last homework (correction)

\begin{problem}[28.1-2]
  题目。
\end{problem}

\begin{cause}
  简述错误原因(可选)。
\end{cause}

% or use the ``solution'' environment
\begin{revision}
  正确解答。
\end{revision}
%%%%%%%%%%%%%%%%%%%%
\beginfb	% begin the feedback section
%%%%%%%%%%%%%%%%%%%%
\end{document}